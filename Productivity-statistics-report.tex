% Options for packages loaded elsewhere
\PassOptionsToPackage{unicode}{hyperref}
\PassOptionsToPackage{hyphens}{url}
%
\documentclass[
]{article}
\usepackage{amsmath,amssymb}
\usepackage{iftex}
\ifPDFTeX
  \usepackage[T1]{fontenc}
  \usepackage[utf8]{inputenc}
  \usepackage{textcomp} % provide euro and other symbols
\else % if luatex or xetex
  \usepackage{unicode-math} % this also loads fontspec
  \defaultfontfeatures{Scale=MatchLowercase}
  \defaultfontfeatures[\rmfamily]{Ligatures=TeX,Scale=1}
\fi
\usepackage{lmodern}
\ifPDFTeX\else
  % xetex/luatex font selection
\fi
% Use upquote if available, for straight quotes in verbatim environments
\IfFileExists{upquote.sty}{\usepackage{upquote}}{}
\IfFileExists{microtype.sty}{% use microtype if available
  \usepackage[]{microtype}
  \UseMicrotypeSet[protrusion]{basicmath} % disable protrusion for tt fonts
}{}
\makeatletter
\@ifundefined{KOMAClassName}{% if non-KOMA class
  \IfFileExists{parskip.sty}{%
    \usepackage{parskip}
  }{% else
    \setlength{\parindent}{0pt}
    \setlength{\parskip}{6pt plus 2pt minus 1pt}}
}{% if KOMA class
  \KOMAoptions{parskip=half}}
\makeatother
\usepackage{xcolor}
\usepackage[margin=1in]{geometry}
\usepackage{color}
\usepackage{fancyvrb}
\newcommand{\VerbBar}{|}
\newcommand{\VERB}{\Verb[commandchars=\\\{\}]}
\DefineVerbatimEnvironment{Highlighting}{Verbatim}{commandchars=\\\{\}}
% Add ',fontsize=\small' for more characters per line
\usepackage{framed}
\definecolor{shadecolor}{RGB}{248,248,248}
\newenvironment{Shaded}{\begin{snugshade}}{\end{snugshade}}
\newcommand{\AlertTok}[1]{\textcolor[rgb]{0.94,0.16,0.16}{#1}}
\newcommand{\AnnotationTok}[1]{\textcolor[rgb]{0.56,0.35,0.01}{\textbf{\textit{#1}}}}
\newcommand{\AttributeTok}[1]{\textcolor[rgb]{0.13,0.29,0.53}{#1}}
\newcommand{\BaseNTok}[1]{\textcolor[rgb]{0.00,0.00,0.81}{#1}}
\newcommand{\BuiltInTok}[1]{#1}
\newcommand{\CharTok}[1]{\textcolor[rgb]{0.31,0.60,0.02}{#1}}
\newcommand{\CommentTok}[1]{\textcolor[rgb]{0.56,0.35,0.01}{\textit{#1}}}
\newcommand{\CommentVarTok}[1]{\textcolor[rgb]{0.56,0.35,0.01}{\textbf{\textit{#1}}}}
\newcommand{\ConstantTok}[1]{\textcolor[rgb]{0.56,0.35,0.01}{#1}}
\newcommand{\ControlFlowTok}[1]{\textcolor[rgb]{0.13,0.29,0.53}{\textbf{#1}}}
\newcommand{\DataTypeTok}[1]{\textcolor[rgb]{0.13,0.29,0.53}{#1}}
\newcommand{\DecValTok}[1]{\textcolor[rgb]{0.00,0.00,0.81}{#1}}
\newcommand{\DocumentationTok}[1]{\textcolor[rgb]{0.56,0.35,0.01}{\textbf{\textit{#1}}}}
\newcommand{\ErrorTok}[1]{\textcolor[rgb]{0.64,0.00,0.00}{\textbf{#1}}}
\newcommand{\ExtensionTok}[1]{#1}
\newcommand{\FloatTok}[1]{\textcolor[rgb]{0.00,0.00,0.81}{#1}}
\newcommand{\FunctionTok}[1]{\textcolor[rgb]{0.13,0.29,0.53}{\textbf{#1}}}
\newcommand{\ImportTok}[1]{#1}
\newcommand{\InformationTok}[1]{\textcolor[rgb]{0.56,0.35,0.01}{\textbf{\textit{#1}}}}
\newcommand{\KeywordTok}[1]{\textcolor[rgb]{0.13,0.29,0.53}{\textbf{#1}}}
\newcommand{\NormalTok}[1]{#1}
\newcommand{\OperatorTok}[1]{\textcolor[rgb]{0.81,0.36,0.00}{\textbf{#1}}}
\newcommand{\OtherTok}[1]{\textcolor[rgb]{0.56,0.35,0.01}{#1}}
\newcommand{\PreprocessorTok}[1]{\textcolor[rgb]{0.56,0.35,0.01}{\textit{#1}}}
\newcommand{\RegionMarkerTok}[1]{#1}
\newcommand{\SpecialCharTok}[1]{\textcolor[rgb]{0.81,0.36,0.00}{\textbf{#1}}}
\newcommand{\SpecialStringTok}[1]{\textcolor[rgb]{0.31,0.60,0.02}{#1}}
\newcommand{\StringTok}[1]{\textcolor[rgb]{0.31,0.60,0.02}{#1}}
\newcommand{\VariableTok}[1]{\textcolor[rgb]{0.00,0.00,0.00}{#1}}
\newcommand{\VerbatimStringTok}[1]{\textcolor[rgb]{0.31,0.60,0.02}{#1}}
\newcommand{\WarningTok}[1]{\textcolor[rgb]{0.56,0.35,0.01}{\textbf{\textit{#1}}}}
\usepackage{graphicx}
\makeatletter
\def\maxwidth{\ifdim\Gin@nat@width>\linewidth\linewidth\else\Gin@nat@width\fi}
\def\maxheight{\ifdim\Gin@nat@height>\textheight\textheight\else\Gin@nat@height\fi}
\makeatother
% Scale images if necessary, so that they will not overflow the page
% margins by default, and it is still possible to overwrite the defaults
% using explicit options in \includegraphics[width, height, ...]{}
\setkeys{Gin}{width=\maxwidth,height=\maxheight,keepaspectratio}
% Set default figure placement to htbp
\makeatletter
\def\fps@figure{htbp}
\makeatother
\setlength{\emergencystretch}{3em} % prevent overfull lines
\providecommand{\tightlist}{%
  \setlength{\itemsep}{0pt}\setlength{\parskip}{0pt}}
\setcounter{secnumdepth}{-\maxdimen} % remove section numbering
\ifLuaTeX
  \usepackage{selnolig}  % disable illegal ligatures
\fi
\IfFileExists{bookmark.sty}{\usepackage{bookmark}}{\usepackage{hyperref}}
\IfFileExists{xurl.sty}{\usepackage{xurl}}{} % add URL line breaks if available
\urlstyle{same}
\hypersetup{
  pdftitle={Analysis of Productivity Stats in NZ from 1978 - 2022},
  hidelinks,
  pdfcreator={LaTeX via pandoc}}

\title{Analysis of Productivity Stats in NZ from 1978 - 2022}
\author{}
\date{\vspace{-2.5em}2024-01-18}

\begin{document}
\maketitle

\hypertarget{analysis-of-productivity-stats-in-nz-from-1978---2022}{%
\subsection{Analysis of Productivity Stats in NZ from 1978 -
2022}\label{analysis-of-productivity-stats-in-nz-from-1978---2022}}

This is a report to investigate the productivity statistics that were
provided by
\href{https://www.stats.govt.nz/large-datasets/csv-files-for-download/}{Stats
NZ}. I've chosen to look through the data to find any meaningful
insights as to how the different industries have changed over the 44
years of data.

\#Introduction

A glance into the data indicates a large variety of productivity
measures contained within. The first rows of the data are as follows:

\begin{Shaded}
\begin{Highlighting}[]
\FunctionTok{head}\NormalTok{(data)}
\end{Highlighting}
\end{Shaded}

\begin{verbatim}
##   Series_reference  Period Data_value Status Units Magnitude
## 1     PRDA.S1CAAZI 1978.03       1000  FINAL Index         0
## 2     PRDA.S1CAAZI 1979.03        947  FINAL Index         0
## 3     PRDA.S1CAAZI 1980.03       1054  FINAL Index         0
## 4     PRDA.S1CAAZI 1981.03       1167  FINAL Index         0
## 5     PRDA.S1CAAZI 1982.03       1144  FINAL Index         0
## 6     PRDA.S1CAAZI 1983.03       1164  FINAL Index         0
##                         Subject
## 1 Productivity Statistics - PRD
## 2 Productivity Statistics - PRD
## 3 Productivity Statistics - PRD
## 4 Productivity Statistics - PRD
## 5 Productivity Statistics - PRD
## 6 Productivity Statistics - PRD
##                                              Group    Type
## 1 Productivity Indexes - Industry Level (ANZSIC06) Capital
## 2 Productivity Indexes - Industry Level (ANZSIC06) Capital
## 3 Productivity Indexes - Industry Level (ANZSIC06) Capital
## 4 Productivity Indexes - Industry Level (ANZSIC06) Capital
## 5 Productivity Indexes - Industry Level (ANZSIC06) Capital
## 6 Productivity Indexes - Industry Level (ANZSIC06) Capital
##                            Industry Measure
## 1 Agriculture, Forestry and Fishing   Index
## 2 Agriculture, Forestry and Fishing   Index
## 3 Agriculture, Forestry and Fishing   Index
## 4 Agriculture, Forestry and Fishing   Index
## 5 Agriculture, Forestry and Fishing   Index
## 6 Agriculture, Forestry and Fishing   Index
\end{verbatim}

A summary of the data provided below:

\begin{Shaded}
\begin{Highlighting}[]
\FunctionTok{summary}\NormalTok{(data)}
\end{Highlighting}
\end{Shaded}

\begin{verbatim}
##  Series_reference       Period       Data_value         Status         
##  Length:14792       Min.   :1978   Min.   :  -28.9   Length:14792      
##  Class :character   1st Qu.:1993   1st Qu.:    1.6   Class :character  
##  Mode  :character   Median :2004   Median :  597.5   Mode  :character  
##                     Mean   :2003   Mean   : 1063.8                     
##                     3rd Qu.:2013   3rd Qu.: 1308.0                     
##                     Max.   :2022   Max.   :64878.0                     
##     Units             Magnitude         Subject             Group          
##  Length:14792       Min.   :0.00000   Length:14792       Length:14792      
##  Class :character   1st Qu.:0.00000   Class :character   Class :character  
##  Mode  :character   Median :0.00000   Mode  :character   Mode  :character  
##                     Mean   :0.08924                                        
##                     3rd Qu.:0.00000                                        
##                     Max.   :3.00000                                        
##      Type             Industry           Measure         
##  Length:14792       Length:14792       Length:14792      
##  Class :character   Class :character   Class :character  
##  Mode  :character   Mode  :character   Mode  :character  
##                                                          
##                                                          
## 
\end{verbatim}

The structure of the data:

\begin{Shaded}
\begin{Highlighting}[]
\FunctionTok{str}\NormalTok{(data)}
\end{Highlighting}
\end{Shaded}

\begin{verbatim}
## 'data.frame':    14792 obs. of  11 variables:
##  $ Series_reference: chr  "PRDA.S1CAAZI" "PRDA.S1CAAZI" "PRDA.S1CAAZI" "PRDA.S1CAAZI" ...
##  $ Period          : num  1978 1979 1980 1981 1982 ...
##  $ Data_value      : num  1000 947 1054 1167 1144 ...
##  $ Status          : chr  "FINAL" "FINAL" "FINAL" "FINAL" ...
##  $ Units           : chr  "Index" "Index" "Index" "Index" ...
##  $ Magnitude       : int  0 0 0 0 0 0 0 0 0 0 ...
##  $ Subject         : chr  "Productivity Statistics - PRD" "Productivity Statistics - PRD" "Productivity Statistics - PRD" "Productivity Statistics - PRD" ...
##  $ Group           : chr  "Productivity Indexes - Industry Level (ANZSIC06)" "Productivity Indexes - Industry Level (ANZSIC06)" "Productivity Indexes - Industry Level (ANZSIC06)" "Productivity Indexes - Industry Level (ANZSIC06)" ...
##  $ Type            : chr  "Capital" "Capital" "Capital" "Capital" ...
##  $ Industry        : chr  "Agriculture, Forestry and Fishing" "Agriculture, Forestry and Fishing" "Agriculture, Forestry and Fishing" "Agriculture, Forestry and Fishing" ...
##  $ Measure         : chr  "Index" "Index" "Index" "Index" ...
\end{verbatim}

The initial year of 1978 was considered the base year with all
industries getting a measure of their productivity as 1000. Below is a
look at how the industries fared in 2022:

\begin{Shaded}
\begin{Highlighting}[]
\NormalTok{prod\_data\_2022 }\OtherTok{\textless{}{-}}\NormalTok{ data[data}\SpecialCharTok{$}\NormalTok{Period }\SpecialCharTok{==} \StringTok{"2022.03"} \SpecialCharTok{\&}\NormalTok{ data}\SpecialCharTok{$}\NormalTok{Measure }\SpecialCharTok{==} \StringTok{"Index"} \SpecialCharTok{\&}\NormalTok{ data}\SpecialCharTok{$}\NormalTok{Type }\SpecialCharTok{==} \StringTok{"Total"}\NormalTok{,]}
\FunctionTok{library}\NormalTok{(ggplot2)}

\CommentTok{\# Will plot the top 5 and bottom 5 industries}
\FunctionTok{ggplot}\NormalTok{(}\AttributeTok{data =}\NormalTok{ prod\_data\_2022, }\FunctionTok{aes}\NormalTok{(}\AttributeTok{x =}\NormalTok{ Data\_value, }\AttributeTok{y =}\NormalTok{ Industry)) }\SpecialCharTok{+}
  \FunctionTok{geom\_bar}\NormalTok{(}\AttributeTok{stat =} \StringTok{"identity"}\NormalTok{, }\AttributeTok{fill =} \StringTok{"skyblue"}\NormalTok{, }\AttributeTok{color =} \StringTok{"black"}\NormalTok{) }\SpecialCharTok{+}
  \FunctionTok{labs}\NormalTok{(}\AttributeTok{title =} \StringTok{"Productivity of all industries in 2022"}\NormalTok{, }\AttributeTok{x =} \StringTok{"Productivity measure"}\NormalTok{, }\AttributeTok{y =} \StringTok{"Industry"}\NormalTok{) }\SpecialCharTok{+}
  \FunctionTok{theme\_minimal}\NormalTok{()}
\end{Highlighting}
\end{Shaded}

\includegraphics{Productivity-statistics-report_files/figure-latex/unnamed-chunk-4-1.pdf}

Taking a closer look into the top 5 most productive industries - which
by the design of the data, have had the highest percentage change since
inception of the measure in NZ.

\begin{Shaded}
\begin{Highlighting}[]
\NormalTok{ordered\_2022\_data }\OtherTok{\textless{}{-}}\NormalTok{ prod\_data\_2022[}\FunctionTok{order}\NormalTok{(prod\_data\_2022}\SpecialCharTok{$}\NormalTok{Data\_value),]}
\NormalTok{top\_5\_industries }\OtherTok{\textless{}{-}}\NormalTok{ ordered\_2022\_data[}\SpecialCharTok{{-}}\NormalTok{(}\DecValTok{1}\SpecialCharTok{:}\DecValTok{17}\NormalTok{),]}
\FunctionTok{ggplot}\NormalTok{(}\AttributeTok{data =}\NormalTok{ top\_5\_industries, }\FunctionTok{aes}\NormalTok{(}\AttributeTok{x =}\NormalTok{ Data\_value, }\AttributeTok{y =}\NormalTok{ Industry)) }\SpecialCharTok{+}
  \FunctionTok{geom\_bar}\NormalTok{(}\AttributeTok{stat =} \StringTok{"identity"}\NormalTok{, }\AttributeTok{fill =} \StringTok{"skyblue"}\NormalTok{, }\AttributeTok{color =} \StringTok{"black"}\NormalTok{) }\SpecialCharTok{+}
  \FunctionTok{labs}\NormalTok{(}\AttributeTok{title =} \StringTok{"Productivity of top 5 industries in 2022"}\NormalTok{, }\AttributeTok{x =} \StringTok{"Productivity measure"}\NormalTok{, }\AttributeTok{y =} \StringTok{"Industry"}\NormalTok{) }\SpecialCharTok{+}
  \FunctionTok{theme\_minimal}\NormalTok{()}
\end{Highlighting}
\end{Shaded}

\includegraphics{Productivity-statistics-report_files/figure-latex/unnamed-chunk-5-1.pdf}

Now looking into the lowest productive industries:

\begin{Shaded}
\begin{Highlighting}[]
\NormalTok{bottom\_5\_industries }\OtherTok{\textless{}{-}}\NormalTok{ ordered\_2022\_data[}\SpecialCharTok{{-}}\NormalTok{(}\DecValTok{6}\SpecialCharTok{:}\DecValTok{22}\NormalTok{),]}
\FunctionTok{ggplot}\NormalTok{(}\AttributeTok{data =}\NormalTok{ bottom\_5\_industries, }\FunctionTok{aes}\NormalTok{(}\AttributeTok{x =}\NormalTok{ Data\_value, }\AttributeTok{y =}\NormalTok{ Industry)) }\SpecialCharTok{+}
  \FunctionTok{geom\_bar}\NormalTok{(}\AttributeTok{stat =} \StringTok{"identity"}\NormalTok{, }\AttributeTok{fill =} \StringTok{"skyblue"}\NormalTok{, }\AttributeTok{color =} \StringTok{"black"}\NormalTok{) }\SpecialCharTok{+}
  \FunctionTok{labs}\NormalTok{(}\AttributeTok{title =} \StringTok{"Productivity of bottom 5 industries in 2022"}\NormalTok{, }\AttributeTok{x =} \StringTok{"Productivity measure"}\NormalTok{, }\AttributeTok{y =} \StringTok{"Industry"}\NormalTok{) }\SpecialCharTok{+}
  \FunctionTok{theme\_minimal}\NormalTok{()}
\end{Highlighting}
\end{Shaded}

\includegraphics{Productivity-statistics-report_files/figure-latex/unnamed-chunk-6-1.pdf}
Having a look into the types of productivity in the top 5 industries:

\begin{Shaded}
\begin{Highlighting}[]
\NormalTok{top\_5\_industries\_by\_name }\OtherTok{\textless{}{-}}\NormalTok{ top\_5\_industries[,}\StringTok{"Industry"}\NormalTok{]}
\NormalTok{all\_industry\_prod\_data\_2022 }\OtherTok{\textless{}{-}}\NormalTok{ data[data}\SpecialCharTok{$}\NormalTok{Period }\SpecialCharTok{==} \StringTok{"2022.03"} \SpecialCharTok{\&}\NormalTok{ data}\SpecialCharTok{$}\NormalTok{Measure }\SpecialCharTok{==} \StringTok{"Index"} \SpecialCharTok{\&}\NormalTok{ data}\SpecialCharTok{$}\NormalTok{Group }\SpecialCharTok{==} \StringTok{"Productivity Indexes {-} Industry Level (ANZSIC06)"}\NormalTok{,]}
\NormalTok{top\_5\_industry\_prod\_data\_2022 }\OtherTok{\textless{}{-}}\NormalTok{ all\_industry\_prod\_data\_2022[all\_industry\_prod\_data\_2022}\SpecialCharTok{$}\NormalTok{Industry }\SpecialCharTok{\%in\%}\NormalTok{  top\_5\_industries\_by\_name,]}

\FunctionTok{ggplot}\NormalTok{(}\AttributeTok{data =}\NormalTok{ top\_5\_industry\_prod\_data\_2022, }\FunctionTok{aes}\NormalTok{(}\AttributeTok{x =}\NormalTok{ Data\_value, }\AttributeTok{y =}\NormalTok{ Industry, }\AttributeTok{fill=}\NormalTok{Type)) }\SpecialCharTok{+}
  \FunctionTok{geom\_bar}\NormalTok{(}\AttributeTok{stat =} \StringTok{"identity"}\NormalTok{, }\AttributeTok{color =} \StringTok{"black"}\NormalTok{) }\SpecialCharTok{+}
  \FunctionTok{labs}\NormalTok{(}\AttributeTok{title =} \StringTok{"Productivity of bottom 5 industries in 2022"}\NormalTok{, }\AttributeTok{x =} \StringTok{"Productivity measure"}\NormalTok{, }\AttributeTok{y =} \StringTok{"Industry"}\NormalTok{) }\SpecialCharTok{+}
  \FunctionTok{theme\_minimal}\NormalTok{()}
\end{Highlighting}
\end{Shaded}

\includegraphics{Productivity-statistics-report_files/figure-latex/unnamed-chunk-7-1.pdf}
Now looking at the bottom 5 industries

\begin{Shaded}
\begin{Highlighting}[]
\NormalTok{bottom\_5\_industries\_by\_name }\OtherTok{\textless{}{-}}\NormalTok{ bottom\_5\_industries[,}\StringTok{"Industry"}\NormalTok{]}
\NormalTok{bottom\_5\_industry\_prod\_data\_2022 }\OtherTok{\textless{}{-}}\NormalTok{ all\_industry\_prod\_data\_2022[all\_industry\_prod\_data\_2022}\SpecialCharTok{$}\NormalTok{Industry }\SpecialCharTok{\%in\%}\NormalTok{  bottom\_5\_industries\_by\_name,]}

\FunctionTok{ggplot}\NormalTok{(}\AttributeTok{data =}\NormalTok{ bottom\_5\_industry\_prod\_data\_2022, }\FunctionTok{aes}\NormalTok{(}\AttributeTok{x =}\NormalTok{ Data\_value, }\AttributeTok{y =}\NormalTok{ Industry, }\AttributeTok{fill=}\NormalTok{Type)) }\SpecialCharTok{+}
  \FunctionTok{geom\_bar}\NormalTok{(}\AttributeTok{stat =} \StringTok{"identity"}\NormalTok{, }\AttributeTok{color =} \StringTok{"black"}\NormalTok{) }\SpecialCharTok{+}
  \FunctionTok{labs}\NormalTok{(}\AttributeTok{title =} \StringTok{"Productivity of bottom 5 industries in 2022"}\NormalTok{, }\AttributeTok{x =} \StringTok{"Productivity measure"}\NormalTok{, }\AttributeTok{y =} \StringTok{"Industry"}\NormalTok{) }\SpecialCharTok{+}
  \FunctionTok{theme\_minimal}\NormalTok{()}
\end{Highlighting}
\end{Shaded}

\includegraphics{Productivity-statistics-report_files/figure-latex/unnamed-chunk-8-1.pdf}

\begin{Shaded}
\begin{Highlighting}[]
\NormalTok{investigating\_data\_group }\OtherTok{\textless{}{-}}\NormalTok{ data[data}\SpecialCharTok{$}\NormalTok{Period }\SpecialCharTok{==} \StringTok{"2022.03"} \SpecialCharTok{\&}\NormalTok{ data}\SpecialCharTok{$}\NormalTok{Measure }\SpecialCharTok{==} \StringTok{"Index"}\NormalTok{,]}
\CommentTok{\#Focus on productivity output}
\NormalTok{prod\_output\_2022 }\OtherTok{\textless{}{-}}\NormalTok{ investigating\_data\_group[investigating\_data\_group}\SpecialCharTok{$}\NormalTok{Group }\SpecialCharTok{==} \StringTok{"Productivity Output Series {-} Industry Level (ANZSIC06)"}\NormalTok{,]}

\FunctionTok{ggplot}\NormalTok{(}\AttributeTok{data =}\NormalTok{ prod\_output\_2022, }\FunctionTok{aes}\NormalTok{(}\AttributeTok{x =}\NormalTok{ Data\_value, }\AttributeTok{y =}\NormalTok{ Industry)) }\SpecialCharTok{+}
  \FunctionTok{geom\_bar}\NormalTok{(}\AttributeTok{stat =} \StringTok{"identity"}\NormalTok{, }\AttributeTok{fill =} \StringTok{"skyblue"}\NormalTok{, }\AttributeTok{color =} \StringTok{"black"}\NormalTok{) }\SpecialCharTok{+}
  \FunctionTok{labs}\NormalTok{(}\AttributeTok{title =} \StringTok{"Productivity output of industries in 2022"}\NormalTok{, }\AttributeTok{x =} \StringTok{"Productivity measure"}\NormalTok{, }\AttributeTok{y =} \StringTok{"Industry"}\NormalTok{) }\SpecialCharTok{+}
  \FunctionTok{theme\_minimal}\NormalTok{()}
\end{Highlighting}
\end{Shaded}

\includegraphics{Productivity-statistics-report_files/figure-latex/unnamed-chunk-9-1.pdf}

\end{document}
